\documentclass[a4paper,10pt]{article}
\usepackage{amsmath,amssymb,bm}
\usepackage[margin=1in]{geometry}
\setlength{\parindent}{0pt}
\setlength{\parskip}{1em}

\begin{document}

\section*{Mathematical Methods Cheatsheet}

\textbf{Calculus of Single and Multiple Variables}
\begin{itemize}
    \item \textbf{Basic Differentiation}: \( \frac{d}{dx} [x^n] = n x^{n-1} \)
    \item \textbf{Basic Integration}: \( \int x^n dx = \frac{x^{n+1}}{n+1} + C \)
    \item \textbf{Partial Derivatives}: \( \frac{\partial f}{\partial x}, \frac{\partial f}{\partial y} \)
    \item \textbf{Exact vs. Inexact Differentials}: \( df \text{ is exact if } \frac{\partial M}{\partial y} = \frac{\partial N}{\partial x} \text{ for } Mdx + Ndy = 0 \)
    \item \textbf{Jacobian Determinant}: \( J = \begin{vmatrix} \frac{\partial x}{\partial u} & \frac{\partial x}{\partial v} \\
    \frac{\partial y}{\partial u} & \frac{\partial y}{\partial v} \end{vmatrix} \)
    \item \textbf{Taylor Expansion}: \( f(x) = f(a) + f'(a)(x-a) + \frac{f''(a)}{2!}(x-a)^2 + \dots \)
    \item \textbf{Fourier Series (Periodic Function \(f(x)\))}: 
    \[ f(x) = a_0 + \sum_{n=1}^{\infty} \left( a_n \cos\frac{n\pi x}{L} + b_n \sin\frac{n\pi x}{L} \right) \]
    where \( a_0 = \frac{1}{T} \int_{-T/2}^{T/2} f(x)dx \), \( a_n = \frac{2}{T} \int_{-T/2}^{T/2} f(x)\cos\frac{n\pi x}{L} dx \), and \( b_n = \frac{2}{T} \int_{-T/2}^{T/2} f(x)\sin\frac{n\pi x}{L} dx \).
    \item \textbf{Tips and Tricks}:
    \begin{itemize}
        \item \textbf{Chain Rule Shortcut}: \( \frac{d}{dx}[f(g(x))] = f'(g(x)) \cdot g'(x) \). Example: \( \frac{d}{dx}[\sin(x^2)] = \cos(x^2) \cdot 2x \).
        \item \textbf{Small Angle Approximations}: For small \( x \), \( \sin(x) \approx x \), \( \cos(x) \approx 1 - \frac{x^2}{2} \), and \( e^x \approx 1 + x \).
    \end{itemize}
\end{itemize}

\textbf{Vector Algebra and Calculus}
\begin{itemize}
    \item \textbf{Vector Dot Product}: \( \mathbf{A} \cdot \mathbf{B} = |\mathbf{A}| |\mathbf{B}| \cos\theta \)
    \item \textbf{Vector Cross Product}: \( \mathbf{A} \times \mathbf{B} = |\mathbf{A}| |\mathbf{B}| \sin\theta \hat{n} \)
    \item \textbf{Gradient}: \( \nabla f = \left( \frac{\partial f}{\partial x}, \frac{\partial f}{\partial y}, \frac{\partial f}{\partial z} \right) \)
    \item \textbf{Divergence}: \( \nabla \cdot \mathbf{A} = \frac{\partial A_x}{\partial x} + \frac{\partial A_y}{\partial y} + \frac{\partial A_z}{\partial z} \)
    \item \textbf{Curl}: \( \nabla \times \mathbf{A} = \begin{vmatrix} \hat{i} & \hat{j} & \hat{k} \\
    \frac{\partial}{\partial x} & \frac{\partial}{\partial y} & \frac{\partial}{\partial z} \\
    A_x & A_y & A_z \end{vmatrix} \)
    \item \textbf{Green's Theorem}: \( \int_C \mathbf{F} \cdot d\mathbf{r} = \iint_R (\nabla \times \mathbf{F}) \cdot \hat{n} \ dA \)
    \item \textbf{Stokes' Theorem}: \( \int_C \mathbf{F} \cdot d\mathbf{r} = \iint_S (\nabla \times \mathbf{F}) \cdot \mathbf{n} \, dS \)
    \item \textbf{Divergence Theorem}: \( \iiint_V \nabla \cdot \mathbf{F} \, dV = \iint_S \mathbf{F} \cdot \mathbf{n} \, dS \)
    \item \textbf{Multiple Integrals}: Evaluate volume integrals as nested integrals, e.g., \( \int_{x_1}^{x_2} \int_{y_1}^{y_2} \int_{z_1}^{z_2} f(x,y,z) dz \ dy \ dx \).
    \item \textbf{Tips and Tricks}:
    \begin{itemize}
        \item \textbf{Gradient Insight}: \( \nabla f \) points in the direction of steepest ascent. Level curves are orthogonal to it.
        \item \textbf{Divergence Theorem Symmetry}: For symmetric fields, use spherical or cylindrical coordinates to simplify volume integrals.
    \end{itemize}
\end{itemize}

\textbf{Differential Equations}
\begin{itemize}
    \item \textbf{First Order Linear ODE}: \( \frac{dy}{dx} + P(x)y = Q(x) \)
    \[ \text{Solution: } y \cdot \mu(x) = \int Q(x) \cdot \mu(x) dx, \text{ where } \mu(x) = e^{\int P(x)dx} \]
    \item \textbf{Second Order Linear ODE (Constant Coefficients)}:
    \[ a\frac{d^2y}{dx^2} + b\frac{dy}{dx} + cy = 0 \]
    \text{Solution depends on roots of characteristic equation } \( ar^2 + br + c = 0 \)
    \begin{itemize}
        \item Distinct Roots: \( y = c_1e^{r_1x} + c_2e^{r_2x} \)
        \item Repeated Roots: \( y = (c_1 + c_2x)e^{r_1x} \)
        \item Complex Roots: \( y = e^{\alpha x} (c_1\cos\beta x + c_2\sin\beta x) \)
    \end{itemize}
    \item \textbf{Homogeneous Equations}: \( Mdx + Ndy = 0 \) is solved using substitution \( y = vx \).
    \item \textbf{Tips and Tricks}:
    \begin{itemize}
        \item \textbf{Variable Separation}: For \( \frac{dy}{dx} = g(y)/h(x) \), separate variables and integrate.
        \item \textbf{Particular Solution Guessing}: Match the form of the non-homogeneous term (e.g., exponential, polynomial, or sine/cosine).
    \end{itemize}
\end{itemize}

\textbf{Matrices and Determinants}
\begin{itemize}
    \item \textbf{Matrix Multiplication}: \( (AB)_{ij} = \sum_k A_{ik}B_{kj} \)
    \item \textbf{Determinant of 2x2 Matrix}: \( \begin{vmatrix} a & b \\ c & d \end{vmatrix} = ad - bc \)
    \item \textbf{Inverse Matrix (2x2)}: \( A^{-1} = \frac{1}{\det(A)} \begin{pmatrix} d & -b \\ -c & a \end{pmatrix} \)
    \item \textbf{Eigenvalues}: Solve \( \det(A - \lambda I) = 0 \) for eigenvalues \( \lambda \).
    \item \textbf{Diagonalization}: For \( A = PDP^{-1} \), where \( D \) is diagonal and columns of \( P \) are eigenvectors.
    \item \textbf{Trace and Determinant}: \( \text{Tr}(A) = \sum \text{eigenvalues}, \det(A) = \prod \text{eigenvalues} \).
    \item \textbf{Tips and Tricks}:
    \begin{itemize}
        \item \textbf{Row Reduction for Determinants}: Use elementary row operations to simplify calculation.
        \item \textbf{Symmetric Matrices}: Eigenvalues are real, and eigenvectors are orthogonal.
        \item \textbf{Complex Matrices}: For Hermitian matrices, eigenvalues are real.
    \end{itemize}
\end{itemize}

\end{document}
