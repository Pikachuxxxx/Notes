\documentclass{article}
\usepackage{amsmath}
\usepackage{listings}
\usepackage{xcolor}
\usepackage{hyperref}
\usepackage{geometry}
\geometry{margin=1in}
\setlength{\parskip}{1em}
\setlength{\parindent}{0em}

\title{Mathematical Examples for Monte Carlo and Quasi-Random Sampling}
\author{}
\date{}

\lstset{
  language=Python,
  basicstyle=\ttfamily\footnotesize,
  keywordstyle=\bfseries\color{blue},
  stringstyle=\color{red},
  commentstyle=\itshape\color{green!50!black},
  frame=single,
  showstringspaces=false,
  breaklines=true,
}

\begin{document}

\maketitle

\section*{Introduction}
This document provides mathematical examples of integration using Monte Carlo sampling and quasi-random sequences. Code snippets in Python are included to demonstrate these techniques, leveraging the matplotlib library for visualization.

\section*{1. Monte Carlo Integration}
Monte Carlo integration estimates the value of an integral by sampling points randomly and averaging the function values.

 Example: Integrating a 1D Function
We aim to approximate the integral:
\[ I = \int_0^1 x^2 dx. \]

Using Monte Carlo sampling:
\[ I \approx \frac{1}{N} \sum_{i=1}^N f(x_i), \]
where \( x_i \) are uniformly sampled points in \([0, 1]\).

 Python Code:
\begin{lstlisting}[caption=Monte Carlo Integration for \(x^2\)]
import numpy as np
import matplotlib.pyplot as plt

# Define the function to integrate
def f(x):
    return x ** 2

# Number of samples
N = 1000

# Monte Carlo sampling
samples = np.random.uniform(0, 1, N)
integral = np.mean(f(samples))

# True value of the integral
true_value = 1 / 3

print(f"Monte Carlo Estimate: {integral}")
print(f"True Value: {true_value}")

# Visualization
x = np.linspace(0, 1, 100)
plt.plot(x, f(x), label='x^2')
plt.scatter(samples, f(samples), color='red', s=1, label='Samples')
plt.legend()
plt.title('Monte Carlo Sampling')
plt.show()
\end{lstlisting}

\section*{2. Quasi-Random Integration Using Halton Sequence}
Quasi-random sequences like Halton reduce variance in integration by distributing samples more evenly.

 Example: Integrating the Same Function with Halton
We generate a Halton sequence for \([0, 1]\) and compute the integral.

 Python Code:
\begin{lstlisting}[caption=Integration with Halton Sequence]
from scipy.stats.qmc import Halton

# Number of samples
N = 1000

# Generate Halton sequence
halton = Halton(d=1)  # 1D sequence
samples = halton.random(n=N).flatten()

# Compute the integral
integral = np.mean(f(samples))

print(f"Halton Estimate: {integral}")
print(f"True Value: {true_value}")

# Visualization
plt.plot(x, f(x), label='x^2')
plt.scatter(samples, f(samples), color='red', s=1, label='Halton Samples')
plt.legend()
plt.title('Halton Sampling')
plt.show()
\end{lstlisting}

\section*{3. Comparing Monte Carlo and Quasi-Random}

 Visualizing Sample Distribution
Compare the distribution of Monte Carlo and Halton samples to see the difference in coverage.

Python Code:
\begin{lstlisting}[caption=Comparing Sample Distributions]
# Generate random and Halton samples
random_samples = np.random.uniform(0, 1, N)
halton_samples = Halton(d=1).random(n=N).flatten()

# Plot
plt.figure(figsize=(12, 6))

plt.subplot(1, 2, 1)
plt.hist(random_samples, bins=30, color='blue', alpha=0.7, label='Random')
plt.title('Monte Carlo Samples')
plt.legend()

plt.subplot(1, 2, 2)
plt.hist(halton_samples, bins=30, color='orange', alpha=0.7, label='Halton')
plt.title('Halton Samples')
plt.legend()

plt.tight_layout()
plt.show()
\end{lstlisting}

\section*{Conclusion}
Monte Carlo integration is versatile and easy to implement, but quasi-random sequences like Halton improve convergence by reducing variance. These examples highlight their mathematical foundation and practical implementation for 1D integrals.

\end{document}
